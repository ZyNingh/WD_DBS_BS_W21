\documentclass[10pt,headsepline=true,toc=flat,paper=a4,DIV=10]{scrartcl}
\usepackage[T1]{fontenc}
\usepackage[utf8]{inputenc}

\usepackage{xspace}
\usepackage{xifthen}
\usepackage{dlfltxbcodetips}
\usepackage{enumitem}

\usepackage{amssymb}
\usepackage{amsmath}
\usepackage{amsthm}
\usepackage{thmtools}
\usepackage{mathrsfs}
\usepackage{mathtools}
\mathtoolsset{mathic}
\usepackage{stmaryrd}
\usepackage{nth}

\usepackage{microtype}
\usepackage{graphicx}

\usepackage{etoolbox}

\usepackage{tikz}
\usetikzlibrary{arrows,decorations.markings,chains,calc,matrix}
\usepackage{tikz-cd}
\tikzset{>=cm to}

\usepackage[colorlinks=false,bookmarks,hidelinks]{hyperref}
\usepackage[all]{hypcap}

\usepackage{csquotes}
\usepackage[english]{babel}
\usepackage{wasysym}

\usepackage[lining]{libertine}
\usepackage[small]{eulervm}

\usepackage{xparse}
\DeclareDocumentCommand{\faktor}{s m O{0.5} m O{-0.5}}{% \newfaktor[*]{#2}[#3]{#4}[#5] -> #2/#4
  \setbox0=\hbox{\ensuremath{#2}}% Store numerator
  \setbox1=\hbox{\ensuremath{\diagup}}% Store slash /
  \setbox2=\hbox{\ensuremath{#4}}% Store denominator
  \raisebox{#3\ht1}{\usebox0}% Numerator
  \mkern-3mu\ifthenelse{\equal{#1}{\BooleanTrue}}% Slash /
    {\diagup}% regular \faktor slash
    {\rotatebox{-35}{\rule[#5\ht2]{0.4pt}{-#5\ht2+#3\ht0+\ht0}}}% tilted rule as a slash
  \mkern-5mu%
  \raisebox{#5\ht2}{\usebox2}% Denominator
}

\declaretheoremstyle[spaceabove=\topsep,spacebelow=\topsep,headfont=\normalfont\scshape,notefont=\normalfont\mdseries,notebraces={(}{)},bodyfont=\normalfont,postheadspace=5pt plus 1pt minus 1pt]{scdef}
\declaretheoremstyle[spaceabove=\topsep,spacebelow=\topsep,headfont=\normalfont\scshape,notefont=\normalfont\mdseries,notebraces={(}{)},bodyfont=\itshape,postheadspace=5pt plus 1pt minus 1pt]{scthm}

\declaretheorem[style=scdef,numberwithin=section,   name=Definition,refname={definition,definitions},Refname={Definition,Definitions}]{definition}
\declaretheorem[style=scdef,sharenumber=definition, name=Remark,refname={remark,remarks},Refname={Remark,Remarks}]{remark}
\declaretheorem[style=scdef,sharenumber=definition, name=Example,refname={example,examples},Refname={Example,Examples}]{example}
\declaretheorem[style=scdef,sharenumber=definition, name=Exercise,refname={exercise,exercises},Refname={Exercise,Exercises}]{exercise}

\declaretheorem[style=scthm,sharenumber=definition, name=Theorem,refname={theorem,theorems},Refname={Theorem,Theorems}]{theorem}
\declaretheorem[style=scthm,sharenumber=definition, name=Lemma,refname={lemma,lemmas},Refname={Lemma,Lemmas}]{lemma}
\declaretheorem[style=scthm,sharenumber=definition, name=Corollary,refname={corollary,corollaries},Refname={Corollary,Corollaries}]{corollary}
\declaretheorem[style=scthm,sharenumber=definition, name=Proposition,refname={proposition,propositions},Refname={Proposition,Propositions}]{proposition}

\undef\Re
\undef\Im
\DeclareMathOperator{\im}{im}
\DeclareMathOperator{\coker}{coker}
\DeclareMathOperator{\Char}{char}
\DeclareMathOperator{\Gal}{Gal}
\DeclareMathOperator{\Aut}{Aut}
\DeclareMathOperator{\id}{id}
\DeclareMathOperator{\Maps}{Maps}
\DeclareMathOperator{\Ind}{Ind}
\DeclareMathOperator{\Rad}{Rad}
\DeclareMathOperator{\Nil}{Nil}
\DeclareMathOperator{\Zar}{Zar}
\DeclareMathOperator{\Spec}{Spec}
\DeclareMathOperator{\Max}{Max}
\DeclareMathOperator{\Proj}{Proj}
\DeclareMathOperator{\proj}{proj}
\DeclareMathOperator{\Ob}{Ob}
\DeclareMathOperator{\Mor}{Mor}
\DeclareMathOperator{\Abb}{Abb}
\DeclareMathOperator{\Ann}{Ann}
\DeclareMathOperator{\pd}{pd}
\DeclareMathOperator{\len}{len}
\DeclareMathOperator{\ord}{ord}
\DeclareMathOperator{\ev}{ev}
\DeclareMathOperator{\tr}{tr}
\DeclareMathOperator{\GL}{GL}
\DeclareMathOperator{\SL}{SL}
\DeclareMathOperator{\orth}{O}
\DeclareMathOperator{\SO}{SO}
\DeclareMathOperator{\Fix}{Fix}
\DeclareMathOperator{\Frob}{Frob}
\DeclareMathOperator{\End}{End}
\DeclareMathOperator{\Hom}{Hom}
\DeclareMathOperator{\IntHom}{\mathbf{Hom}}
\DeclareMathOperator{\Re}{Re}
\DeclareMathOperator{\Im}{Im}
\DeclareMathOperator{\dist}{dist}
\DeclareMathOperator{\obj}{Ob}
\DeclareMathOperator{\Ouv}{\mathbf{Ouv}}
\DeclareMathOperator{\PShv}{\mathbf{PShv}}
\DeclareMathOperator{\Shv}{\mathbf{Shv}}
\DeclareMathOperator{\sheaf}{a}
\DeclareMathOperator{\Ho}{Ho}
\DeclareMathOperator{\Sing}{Sing}
\DeclareMathOperator{\cinj}{in}
\DeclareMathOperator{\Deck}{Deck}
\DeclareMathOperator{\diam}{diam}
\DeclareMathOperator{\rk}{rk}
\DeclareMathOperator*{\colim}{colim}
\DeclareMathOperator{\Ch}{\ensuremath{\mathbf{Ch}^+\mkern-2mu}}
\DeclareMathOperator{\Chl}{\ensuremath{\mathbf{Ch}_+\mkern-2mu}}
\DeclareMathOperator{\Dc}{\ensuremath{\mathbf{D}^+\mkern-2mu}}
\DeclareMathOperator{\Tor}{Tor}
\DeclareMathOperator{\Ext}{Ext}
\DeclareMathOperator{\res}{res}
\DeclareMathOperator{\sgn}{sgn}

\newcommand*{\alg}[1]{\ensuremath{\overline{#1}}\xspace}
\newcommand*{\sep}[1]{\ensuremath{#1_{\mathrm{sep}}}\xspace}
\newcommand*{\ab}[1]{\ensuremath{#1^{\mathrm{ab}}}\xspace}
\newcommand*{\units}[1]{\ensuremath{#1^\times}\xspace}
\newcommand*{\dualg}[1]{\ensuremath{#1^*}\xspace}
\newcommand*{\lapl}{\Delta}
\newcommand*{\grad}{\nabla}
\newcommand*{\leng}[1]{\ensuremath{\operatorname{L}(#1)}\xspace}
\newcommand*{\op}[1]{\ensuremath{#1^{\mathrm{op}}}\xspace}
\newcommand*{\codiag}[1]{\ensuremath{\nabla\mkern-3mu_{#1}}\xspace}
\newcommand*{\diag}[1]{\ensuremath{\Delta_{#1}}\xspace}
\newcommand*{\local}[2]{\ensuremath{#1[{#2}^{-1}]}\xspace}
\newcommand*{\inj}[1]{\ensuremath{\operatorname{in}_{#1}}\xspace}
\newcommand*{\Lan}[1]{\operatorname{Lan}_{#1}}
\newcommand*{\Ran}[1]{\operatorname{Ran}_{#1}}
\newcommand*{\Lder}[1]{\mathbf{L}{#1}}
\newcommand*{\Rder}[1]{\mathbf{R}{#1}}
\newcommand*{\TLder}[1]{\mathbb{L}{#1}}
\newcommand*{\TRder}[1]{\mathbb{R}{#1}}
\newcommand*{\pow}[1]{\ca{P}(#1)}
\newcommand*{\normaliser}[1]{\operatorname{N}(#1)}
\newcommand*{\tor}[1]{#1_{\mathrm{tor}}}

\newcommand*{\Vh}{V_{\mathrm{h}}}
\newcommand*{\Dh}{D_{\mathrm{h}}}
\newcommand*{\Hloc}{H^{\text{loc}}}
\newcommand*{\Hcoloc}{H_{\text{loc}}}

\newcommand*{\Mod}[1]{\ensuremath{\text{$#1$--\textbf{Mod}}}}
\newcommand*{\Vect}[1]{\ensuremath{\text{$#1$--\textbf{Vect}}}}
\newcommand*{\RMod}[1]{\ensuremath{\text{\textbf{Mod}--$#1$}}}
\newcommand*{\BiMod}[2]{\ensuremath{\text{$#1$--\textbf{Mod}--$#2$}}}
\newcommand*{\Kom}[1]{\ensuremath{\text{$#1$--\textbf{Kom}}}}
\newcommand*{\Ring}{\mathbf{Ring}}
\newcommand*{\KomRing}{\mathbf{Ring}}
\newcommand*{\Sch}{\mathbf{Sch}}
\newcommand*{\Set}{\mathbf{Set}}
\newcommand*{\Top}{\mathbf{Top}}
\newcommand*{\Rin}{\mathbf{Rin}}
\newcommand*{\Grp}[1][]{\ifthenelse{\isempty{#1}}{\ensuremath{\mathbf{Grp}}}{\ensuremath{\text{$#1$-\textbf{Grp}}}}}
\newcommand*{\Ab}{\mathbf{Ab}}
\newcommand*{\Alg}[1]{\ensuremath{\text{$#1$-\textbf{Alg}}}}
\newcommand*{\HAlg}[1]{\ensuremath{\text{$#1$-\textbf{Hopfalg}}}}
\newcommand*{\one}{\ensuremath{\mathds{1}}}
\newcommand*{\Ord}{\mathbf{\Delta}}
\newcommand*{\CGHaus}{\mathbf{CGHaus}}
\newcommand*{\RP}{\mathbb{RP}}
\newcommand*{\CP}{\mathbb{CP}}
\newcommand*{\simp}[1]{\ensuremath{\mathbf{s}{#1}}\xspace}

\newcommand*{\normal}{\lhd}
\newcommand*{\isom}{\cong}
\newcommand*{\homot}{\simeq}

\makeatletter
\let\@oldsubset=\subset
\def\@subsethelper#1#2{\mathrel{\raisebox{.5pt}{$#1\@oldsubset$}}\xspace}
\DeclareRobustCommand*{\subset}{\mathpalette\@subsethelper\relax}

\let\@oldotimes=\otimes
\def\@otimeshelper#1#2{\mathrel{\raisebox{.5pt}{$#1\@oldotimes$}}\xspace}
\DeclareRobustCommand*{\otimes}{\mathpalette\@otimeshelper\relax}
\makeatother

\renewcommand*{\setminus}{\smallsetminus}

\makeatletter
\newbox\@xrat
\renewcommand*{\xrightarrow}[2][-cm to]{%
  \setbox\@xrat=\hbox{\ensuremath{\scriptstyle #2}}
  \pgfmathsetlengthmacro{\@xratlen}{max(1.6em,\wd\@xrat+.6em)}
  \pgfmathsetlengthmacro{\@xratinnersep}{.5ex-\dp\@xrat}
  \mathrel{\tikz [#1,baseline=-.6ex]
    \draw (0,0) -- node[auto,inner sep=\@xratinnersep] {\box\@xrat} (\@xratlen,0) ;}}
\renewcommand*{\xleftarrow}[2][cm to-]{%
  \setbox\@xrat=\hbox{\ensuremath{\scriptstyle #2}}
  \pgfmathsetlengthmacro{\@xratlen}{max(1.6em,\wd\@xrat+.6em)}
  \pgfmathsetlengthmacro{\@xratinnersep}{.5ex-\dp\@xrat}
  \mathrel{\tikz [#1,baseline=-.6ex]
    \draw (0,0) -- node[auto,inner sep=\@xratinnersep] {\box\@xrat} (\@xratlen,0) ;}}
\newcommand*{\xrightarrowb}[2][-cm to]{%
  \setbox\@xrat=\hbox{\ensuremath{\scriptstyle #2}}
  \pgfmathsetlengthmacro{\@xratlen}{max(1.6em,\wd\@xrat+.6em)}
  \pgfmathsetlengthmacro{\@xratinnersep}{.5ex}
  \mathrel{\tikz [#1,baseline=-.6ex]
    \draw (0,0) -- node[auto,inner sep=\@xratinnersep] {\box\@xrat} (\@xratlen,0) ;}}
\newcommand*{\xleftarrowb}[2][cm to-]{%
  \setbox\@xrat=\hbox{\ensuremath{\scriptstyle #2}}
  \pgfmathsetlengthmacro{\@xratlen}{max(1.6em,\wd\@xrat+.6em)}
  \pgfmathsetlengthmacro{\@xratinnersep}{.5ex}
  \mathrel{\tikz [#1,baseline=-.6ex]
    \draw (0,0) -- node[auto,inner sep=\@xratinnersep] {\box\@xrat} (\@xratlen,0) ;}}

\pgfarrowsdeclare{my right hook}{my right hook}
{
\arrowsize=0.2pt
\advance\arrowsize by .5\pgflinewidth
\pgfarrowsleftextend{-.5\pgflinewidth}
\pgfarrowsrightextend{3.5\arrowsize+.5\pgflinewidth}
}
{
\arrowsize=0.2pt
\advance\arrowsize by .5\pgflinewidth
\pgfsetdash{}{0pt} % do not dash
\pgfsetroundjoin % fix join
\pgfsetroundcap % fix cap
\pgfpathmoveto{\pgfpoint{0\arrowsize}{-7\arrowsize}}
\pgfpatharc{-90}{90}{3.5\arrowsize}
\pgfusepathqstroke
}

\tikzset{%
  iso/.style={above,sloped,inner sep=0},
  iso'/.style={below,sloped,inner sep=0},
  to/.style={-cm to},
  onto/.style={-cm double to},
  into/.style={my right hook-cm to},
  mapsto/.style={|-cm to}
}

\newcommand*\@tikzto[2]{\mathrel{\begin{tikzpicture}[baseline]%
      \draw[to,line width={#2\pgflinewidth},scale=#1](0,.55ex) -- (1.6em,.55ex);%
    \end{tikzpicture}}}

\newcommand*\@tikzcto[2]{\mathrel{\begin{tikzpicture}[baseline]%
      \draw[to,line width={#2\pgflinewidth},scale=#1](0,.55ex) -- (0.8em,.55ex);%
    \end{tikzpicture}}}

\newcommand*\@tikzonto[2]{\mathrel{\begin{tikzpicture}[baseline]%
      \draw[onto,line width={#2\pgflinewidth},scale=#1](0,.55ex) -- (1.6em,.55ex);%
    \end{tikzpicture}}}

\newcommand*\@tikzinto[2]{\mathrel{\begin{tikzpicture}[baseline]%
      \draw[into,line width={#2\pgflinewidth},scale=#1](0,.55ex) -- (1.6em,.55ex);%
    \end{tikzpicture}}}

\newcommand*\@tikzmapsto[2]{\mathrel{\begin{tikzpicture}[baseline]%
      \draw[mapsto,line width={#2\pgflinewidth},scale=#1](0,.55ex) -- (1.6em,.55ex);%
    \end{tikzpicture}}}

\newcommand*\@tikziso[4]{\mathrel{\begin{tikzpicture}[baseline]%
      \draw[to,line width={#2\pgflinewidth},scale=#1](0,.55ex) -- node[iso,pos=0.47,inner sep=#4]{$#3\sim$} (1.6em,.55ex);%
    \end{tikzpicture}}}

\newcommand*\tikzto{\mathchoice{\@tikzto{1.0}{1}}{\@tikzto{1.0}{1}}{\@tikzcto{0.8}{0.9}}{\@tikzcto{0.6}{0.75}}}
\newcommand*\tikzcto{\mathchoice{\@tikzcto{1.0}{1}}{\@tikzcto{1.0}{1}}{\@tikzcto{0.8}{0.9}}{\@tikzcto{0.6}{0.75}}}
\newcommand*\tikzonto{\mathchoice{\@tikzonto{1.0}{1}}{\@tikzonto{1.0}{1}}{\@tikzonto{0.8}{0.9}}{\@tikzonto{0.6}{0.75}}}
\newcommand*\tikzinto{\mathchoice{\@tikzinto{1.0}{1}}{\@tikzinto{1.0}{1}}{\@tikzinto{0.8}{0.9}}{\@tikzinto{0.6}{0.75}}}
\newcommand*\tikzmapsto{\mathchoice{\@tikzmapsto{1.0}{1}}{\@tikzmapsto{1.0}{1}}{\@tikzmapsto{0.8}{0.9}}{\@tikzmapsto{0.6}{0.75}}}
\newcommand*\tikziso{\mathchoice{\@tikziso{1.0}{1}{\displaystyle}{0pt}}%
  {\@tikziso{1.0}{1}{\textstyle}{0pt}}%
  {\@tikziso{0.8}{0.9}{\scriptstyle}{0pt}}%
  {\@tikziso{0.67}{0.8}{\scriptscriptstyle}{0.15ex}}}

\makeatother

\renewcommand*{\to}[1][]{\ifthenelse{\isempty{#1}}{\tikzto}{\xrightarrowb{#1}}}
\newcommand*{\cto}{\ensuremath{\tikzcto}}
\newcommand*{\into}[1][]{\ifthenelse{\isempty{#1}}{\tikzinto}{\xrightarrowb[into]{#1}}}
\newcommand*{\onto}[1][]{\ifthenelse{\isempty{#1}}{\tikzonto}{\xrightarrowb[onto]{#1}}}

\newcommand*{\iso}{\tikziso}

\renewcommand*{\mapsto}{\tikzmapsto}

\newenvironment{smallpmatrix}{\bigl(\begin{smallmatrix}}{\end{smallmatrix}\bigr)}

\newcommand*{\ca}[1]{\ensuremath{\mathcal{#1}}\xspace}
\renewcommand*{\cal}[1]{\ensuremath{\mathcal{#1}}\xspace}
\newcommand*{\f}[1]{\ensuremath{\mathfrak{#1}}\xspace}

\newcommand*{\dd}{\,\mathrm{d}}

\newcommand{\cl}[2][0]{{}\mkern#1mu\overline{\mkern-#1mu#2}}
\newcommand*{\Int}[1]{\ensuremath{#1^\circ}\xspace}

\newcommand*{\ie}{i.\,e.}
\newcommand*{\eg}{e.\,g.}

\def\<#1>{\left\langle #1 \right\rangle}

\undef\AA
\undef\SS
\renewcommand*{\do}[1]{\expandafter\def\csname#1#1\endcsname{\ensuremath{\mathbb{#1}}\xspace}}
\docsvlist{A,B,C,D,E,F,G,H,I,J,K,L,M,N,O,P,Q,R,S,T,U,V,W,X,Y,Z}

\setlist[enumerate]{label={\upshape(\roman*)}, nosep}

\hyphenation{lo-ca-li-sa-tion}
\hyphenation{pre-com-po-si-tion}
\hyphenation{se-quence}

\overfullrule=1mm
