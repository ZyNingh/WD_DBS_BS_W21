\subsection{Gruppenwirkung}

We just define left group action here. Right group action is quite similar.(Actually we don't use it at all)
\begin{definition}[Left group action]
    Let $G$ be a group with neutral element $e$, and \(X\) is a set. Then (left) group action $s$ of $G$ on $X$ is a function \(s:G\times X \to X\), that satisfied the following two axiom: 
    \begin{itemize}
        \item $s(e,x) = x$
        \item \(s(g,s(h,x)) = s(gh,x)\)
    \end{itemize}
    (with $s(g,x)$ often shortened to $gx$ or $g\cdot x $ when the action being considered is clear from context.)
    \begin{itemize}
        \item $ex = x$
        \item \(g(hx) = (gh)x\)
    \end{itemize}
    for all $g$ and $h$ in $G$ and all $x$ in $X$.
\end{definition}

\begin{remark}
    \begin{itemize}
        \item The group G is said to act on X (from the left). A set X together with an action of G is called a (left) G-set.
        \item You must know the operation in $gh$ and $hx$ is different.
    \end{itemize}
\end{remark}

\begin{definition}[Bahn or orbit]
    Consider a group $G$ acting on a set $X$. The orbit of an element $x$ in $X$ is the set of elements in $X$ to which $x$ can be moved by the elements of $G$. The orbit of $x$ is denoted by $Gx$. $$G\cdot x = \{g\cdot x :g \in G\}$$
\end{definition}

\begin{remark}
    \item  The action is transitive iff it has exactly only one orbit, that is, if there exists $x$ in $X$ with $Gx =X$. This is the case iff $Gx= X$ for all $x$ in $X$.
    \item Two orbit is same or disjoint, because $gx = hy \in Gx\cap Gy\Rightarrow x = g^{-1}hy, also Gx\subset Gy$ and $y = h^{-1}gx, \text{also }  Gy\subset Gx$. So X is disjoint union of every orbit. (i.e. orbits is a parition.)
\end{remark}

Given $g$ in $G$ and $x$ in $X$ with $ g\cdot x=x$, $g\cdot x=x$ , it is said that "$x$ is a fixed point of g" or that "g fixes $x$". Then we can defien stabilizer and fixed points.
\begin{definition}[Stabilizer and fixed points]
    \begin{itemize}
        \item For every $x$ in \(X\), the stabilizer subgroup of $G$ with respect to $x$ is the set of all element in $G$ that fix $x$. $G_x :=\{g\in G\mid gx = x\}$.
        \item $x$ ist fixed point if $Gx = \{x\}$. $X^G$ is set of all fixed(invariant) point.
    \end{itemize}
\end{definition}
\begin{remark}
    The action of $G$ on $X$ is free iff all stabilizer are trivial.
\end{remark}

\begin{lemma} 
    Let $G\times X \to X$ a action of $G$ on $X$. Then \[|X|=\sum_{B\backslash X} |B|\]
\end{lemma}

\begin{lemma} \label{Bkg}
    Let $G\times X \to X$ a action of $G$ on $X$. For every $x\in X$ the function $\varphi : G \to X, g\mapsto gx $ can be reduced to a isomorphism $G/G_x\simeq Gx$, where $G/G_X$ is set of left cosets. In particular it holds $\ord(Gx) = G : Gx$
\end{lemma}
\begin{proof}
    It holds \(\varphi (g) = \varphi (h) \Leftrightarrow gx = hx \Leftrightarrow h^{-1}gx =x \Leftrightarrow h^{-1} g \in G_x \Leftrightarrow gG_x= hG_x\)\\
    Also $\varphi$ is a injective function $G \to X, g\mapsto $. Obvioulsy is $\varphi $ surjective.\\
    Following is something more deteailed.[Fraleigh 16.16 Theorem]
    \begin{itemize}
        \item We define a one-to-one map \(\phi\) from \(Gx\) onto the collection of left costes of \(G_x\) in \(G\). i.e. \(\phi(x_1)=g_1G_x\)
        \item We need to show that the function ist well defined, injective and surjective.(See the book for detail, it is just calculation).
        \item If \(|G|\) is finite, then the equation \(|G| = |G_x|(G:G_x)\) shows that \(|Gx| = (G:G_x)\) is a divisor of \(|G|\).
    \end{itemize}
\end{proof}

\begin{theorem}[Bahnengleicung] \label{Bg}
    $$|X| = \sum_{i=1}^r |Gx_i| = \sum_{i=1}^r |G:G_{x_i}of solvable groups are solv|$$
\end{theorem}
\begin{proof}
    \(X\) is disjoint union of all orbit \(B_i\) with \(B_i = Gx_i\). Then holds \(|X| = \sum_{i=1}^r |Gx_i| = \sum_{i=1}^r |G:G_{x_i}|\) directly by lemma.
\end{proof}

\begin{lemma}
    Let $G\times X \to X$ a action of $G$ on $X$. If \(x\in X,  g\in G\), then \(G_{gx} = gG_xg^{-1}\)
\end{lemma}
\begin{proof}
    \(h\in G_{gx}\Leftrightarrow hgx = gx \Leftrightarrow g^{-1}hgx =x\Leftrightarrow g^{-1}hg \in G_x\Leftrightarrow h\in gG_xg^{-1}\)
\end{proof}

\begin{remark}
    In some skript man can find auch Klassengleichung. It is a special case of Bahnengleichung with respect conjugationfunction and center.
\end{remark}