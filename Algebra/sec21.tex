
\section{Gruppentheorie}
\subsection{Grundbegriffe der Gruppentheorie}


\begin{definition}[Gruppe]
    A group is a set $G$ together with a binary operation on $G$, here denoted "$\cdot$", that combines any two element $a$ and $b$ to from an element of $G$, denoted by $a\cdot b $, such that following three requirements, known as group axiom, are statisfied:\begin{itemize}
        \item Associativity: $\forall a,b \in G: (a\cdot b )\cdot c = a \cdot(b\cdot c)$
        \item Identity element: $\exists e \in G \forall a \in G : a\cdot e = e \cdot a = a$
        \item Inverse element: $\forall a \in G\exists b \in G : a \cdot b = e$
    \end{itemize}
\end{definition}
\begin{remark}
    The definition of a group don't require that $\forall a,b\in G: a \cdot b = b \cdot a$. If this additional condition holds, then the operation is said to be commutative, and the group is called an abelian group.
   
\end{remark}

Following are some basic properties of group. Proof would not be repeated here.

\begin{proposition}
    \begin{itemize}
        \item The neutral element is unique.
        \item Inverse in group is unique.
    \end{itemize}
\end{proposition}

\begin{example}
    The direct product:    Let $G_1,G_2$ be groups. Let $ G_1\times G_2$ be the direct product as sets. We can define the product componentwise by $(x_1, x_2)(y_1,y_2) = (x_1y_1,x_2y_2)$. Then $G_1 \times G_2$ is a group, whose unit element is $(e_1,e_2)$.
\end{example}

\begin{definition}[Subgroup]
    Let $G$ be a group. A subgroup $H$ of $G$ is a subset of $G$ containing the unit element, and such that $H$ is closed under the law of composition and inverse.
\end{definition}
\begin{remark}
    A subgroup is called trivial if it consists of the unit element alone.
\end{remark}

\begin{definition}[Generator]
    Let $G$ be a group and $S$ be a subset of $G$. We shall say that $S$ generates $G$, or that $S$ is a set of generators for $G$, if every element $G$ can be expressed as a product of elements of $S$ or inverses of elements of $S$, i.e. as a product $x_1\cdots x_n$ where each $ x_i $ or $ x_i^{-1}$ is in $S$. 
\end{definition}
\begin{remark}
    \begin{itemize}
        \item It is clear that the set of all such products is subgroup of $G$, and is the smallest subgroup of $G$ containing $S$.
        \item $S$ generates $G$ iff the smallest subgroup of $G$ containing $S$ is $G$ itself. If $G$ is generated by $S$, then we write $G=\langle S\rangle$
        \item By definition, a cyclic group is a group which has one generator.
        \item Given elements $x_1,\cdots, x_n\in G$, these elements generate a subgroup $\langle x_1,\cdots , x_n\rangle$, namely the set of all element of $G$ of the form $$x_{i_1}^{k_1}\cdots x_{i_r}^{k_r}\text{  with  } k_1,\cdots, k_r\in \mathbb{Z}$$
        \item A single element $x\in G$ generates a cyclic subgroup.
    \end{itemize}
\end{remark}

\begin{lemma}
    Let $H$ a nonempty subset of $G$. If $a^{-1}b\in H $ for all $a,b\in H$, $H$ is a subgroup of $G$.
\end{lemma}

\begin{definition}[Grouphomomorphism]
    Let $G,G'$ be groups. A grouphomomorphism of $G$ into $G'$ is a mapping $f:G\to G'$ such that $f(xy) = f(x)f(y)$ for all $x,y\in G$.
\end{definition}
\begin{remark}
    \begin{itemize}
        \item Let $f:G\to G'$ be a grouphomomorphism. Then $f(x^{-1})= f(x)^{-1}$ and $f(e)= e'$.
        \item Composition of homomorphism is homomorphism.
        \item A homomorphism $f:G\to G'$ is called an isomorphism if there exists a homomorphism $g:G' \to G $ such that $f\circ g ,\ g \circ f$ are the identity mapping. Obviously $f$ is isomorphism iff $f$ is bijective. The existence of an isomorphism between two group $G$ and $G'$ is sometimes denoted $G\sim G'$. If $G= G'$ ,we say that isomorphism is an automorphism. A Homomorphism of $G$ into itself is also called an endomorphism.
    \end{itemize}
    
\end{remark}

\begin{definition}[Kernel and image]
    Let $f:G\to G'$ be a grouphomomorphism. Let $e,e'$ be the respective unit element of $G,G'$. We define the kernel of $f$ be the subset of $G$ consisting of all $x$ such that $f(x) = e'$. Let $H'$ be the image of $f$.
\end{definition}
\begin{remark}
    \begin{itemize}
    \item From the definition, it follows at once that the kernel $H$ of $f$ is a subgroup $G$. $H'$ is a normal subgroup of $G'$.
    \item The kernel and image of $f$ are sometimes denoted by $\ker f$ and $\im f$.
    \item A homomorphism whose kernel is trivial is injective.
\end{itemize}
\end{remark}


\begin{definition}[Centralizer]
    Define $C_{G}(A)=\left\{g \in G \mid g a g^{-1}=a\right.$ for all $\left.a \in A\right\}$. This subset of $G$ is called the centralizer of $A$ in $G$. Since $g a g^{-1}=a$ if and only if $g a=a g, C_{G}(A)$ is the set of elements of $G$ which commute with every element of $A$.
\end{definition}
\begin{remark}
    Centralizer is subgroup.
\end{remark}
\begin{definition}[Center]
    Define $Z(G)=\{g \in G \mid g x=x g$ for all $x \in G\}$, the set of elements commuting with all the elements of $G .$ This subset of $G$ is called the center of $G .$
\end{definition}
\begin{remark}
    Note that $Z(G)=C_{G}(G)$, so the argument above proves $Z(G) \leq G$ as a special case.
\end{remark}
\begin{definition}[Normalizer]
    Definition. Define $g A g^{-1}=\left\{g a g^{-1} \mid a \in A\right\}$. Define the normalizer of $A$ in $G$ to be the set $N_{G}(A)=\left\{g \in G \mid g A g^{-1}=A\right\}$.
\end{definition}
\begin{remark}
    Notice that if $g \in C_{G}(A)$, then $g a g^{-1}=a \in A$ for all $a \in A$ so $C_{G}(A) \leq N_{G}(A) .$ 
\end{remark}

\begin{definition}[Coset]
    Let $G$ be a group and $H$ a subgroup. A left coset of $H$ is $G$ is a subset of $G$ og type $aH$ for some element $a$ of $G$.
    \[aH:=\{ab:b\in H\}\]Any element of a coset is called a representative for the coset.
\end{definition}
\begin{lemma}
    Let $N$ be any subgroup of the group $G$. The set of left cosets of $N$ in $G$ form a parition of $G$. Furthermore, for all $u,v\in G$. Furthermore, for all $u,v\in G, uN = vN$ iff  $v^{-1}u \in N$, and in particular, $uN = vN$ iff $u$ and $v$ are representatives of the same coset.
\end{lemma}

\begin{proposition}
   Let $G$ be a group and let $N$ be a subgroup of $G$.\begin{itemize}
    \item The operation on the set of left cosets of $N$ in $G$ described by
    $$
    u N \cdot v N=(u v) N
    $$
    is well defined if and only if $g n g^{-1} \in N$ for all $g \in G$ and all $n \in N$.
    \item  If the above operation is well defined, then it makes the set of left cosets of $N$ in $G$ into a group. In particular the identity of this group is the coset $1 N$ and the inverse of $g N$ is the coset $g^{-1} N$ i.e., $(g N)^{-1}=g^{-1} N$.
   \end{itemize}
\end{proposition}

\begin{definition}[normal]
    The element $g n g^{-1}$ is called the conjugate of $n \in N$ by $g$. The set $g N g^{-1}=\left\{g n g^{-1} \mid n \in N\right\}$ is called the conjugate of $N$ by $g$. The element $g$ is said to normalize $N$ if $g N g^{-1}=N$. A subgroup $N$ of a group $G$ is called normal if every element of $G$ normalizes $N$, i.e., if $g N g^{-1}=N$ for all $g \in G$. If $N$ is a normal subgroup of $G$ we shall write $N \unlhd G$.
\end{definition}


\begin{remark}
    \begin{itemize}
        \item  $
            N\unlhd G
            \Leftrightarrow  \forall a \in G : aNa^{-1} = N
            \Leftrightarrow  \forall a \in G : aNa^{-1} \subset N$\\
            ($\forall g :H = g(g^{-1}Hg)g^{-1}\subset gHg^{-1}\subset H$)
        \item  Aber es gilt \(gHg^{-1} \subset H \not \Rightarrow gHg^{-1} = H \)
    \end{itemize}
\end{remark}

\begin{theorem}
    Let $N$ be a subgroup of the group $G$. The following are equivalent:
    \begin{enumerate}
        \item $N \unlhd G$
        \item $N_G(N) = G$
        \item $gN=Ng$ for all $g\in G$
        \item $gNg^{-1} $ for all $g\in G$.
    \end{enumerate}
\end{theorem}

FIXME:zhe TM sha,shuizhidaotacongnalikaishidingyia
TODO: QUOTIENT GROUP.

Hier sollten die Definition von Faktorgruppen sein. Aber ich weisse nicht wie ich das machen kann.

\begin{definition}[Faktorgruppe]
    
\end{definition}

\begin{theorem}[Fundamental theorem on homomorphisms]
    Let $\phi : G\to G'$ be a grouphomomorphism and $N\unlhd G$ with $N\subset \ker \varphi$. There is a unique grouphomomorphism $$\bar{\varphi}: G/N\to G'$$, such that $$
    \begin{tikzcd}
        G \arrow[rr, "\varphi"] \arrow[rd, "\pi"] &                                 & G' \\
                                                  & G/N \arrow[ru, "\bar{\varphi}"] &   
        \end{tikzcd}$$ commute.
\end{theorem}

In particular $$G/\ker f \simeq \im f \leq G' $$.\\

Proof is egal.

TODO: Also you should know what is order of a element in a group. But I just don't know where should i write that.

\begin{theorem}[Universelle Eigenschaft der Faktorgruppe]
    Let $G$ be a group, $N$ a normal subgroup $N\unlhd G$, $\pi$ canonical projection $\pi: G \to G/N, g \mapsto gN.$ Then $$ \varphi \text{ factorize the quotient group } G/N(\text{i.e. }\exists \bar{\varphi} : G/N \to G' \text{ Homo.}) \Leftrightarrow N \subset \ker \varphi$$.
\end{theorem}
Notice:\[% https://tikzcd.yichuanshen.de/#N4Igdg9gJgpgziAXAbVABwnAlgFyxMJZABgBpiBdUkANwEMAbAVxiRAHEQBfU9TXfIRQAmclVqMWbdgHoAct14gM2PASIBGUhvH1mrRBwDk3cTCgBzeEVAAzAE4QAtkjIgcEJFon62AHT80LBBqBjoAIxgGAAV+NSEQBhhbHEU7RxdENw8kUR8pQwD6ezQAC2DQiKjY1UE2eywLUtSedOcvahzEPL0CkACYAA8sOBw4AAIA8Lp7YCKZsqwuEMSqmLi6wySU0y4gA
\begin{tikzcd}
G \arrow[rr, "\pi"] \arrow[rd, "\varphi"'] &    & G/N \arrow[ld, "\exists \bar{\varphi}"] \\
                                           & G' &                                        
\end{tikzcd}\]

TODO: I think I still need proof of lagrange theorem here. Maybe I do it later.