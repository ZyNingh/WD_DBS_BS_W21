\subsection{Aufl\"osbare Gruppe}
\begin{definition}[Derived series]
    This construction can be iterated:
    \[G^{(0)}:=G\]
    \[G^{(n)}:=[G^{(n-1)},G^{(n-1)}]\] 
    The groups \[G^{(2)},G^{(3)},...\] are called the second derived subgroup, third derived subgroup, and so forth the descending normal series \(\cdots\unlhd G^{(2)}\unlhd G^{(1)}\unlhd G^{(0)} = G\) is called derived series.
\end{definition}

\begin{definition}
    A group is called solvable, if its derived series, the descending normal series, where every subgroup is the commutator subgroup of the previous one, eventually reaches the trivial subgroup of \(G\).
\end{definition}
\begin{remark}
    In many book there is equivalent definition: A group \(G\)  solvable if it has a subnormal series whose factor groups (quotient groups) are all abelian, that is, if there are subgroups $1 = G_0 < G_1 < \cdots < G_k = G$ such that $G_{j-1}$ is normal in $G_j$, and $G_j/G_{j-1}$ is an abelian group, for $j = 1, 2, ..., k$. A proof to this could be found in Dummit Sec. 6.1 Theorem 9.
\end{remark}

\begin{proposition}\label{sol}
    Let \(G\) and \(K\) be groups, let \(H\) be a subgroup of \(G\) and let \(\varphi: G\to K\) be a surjective homomorphism.\begin{itemize}
        \item \(H^{(i)}\leq G^{(i)}\) for all \(i\geq 0\). In particular, if \(G\) is solvable, then so is \(H\), i.e. subgroups of solvable groups are solvable (and the solvable length of $H$ is less than or equal to the solvable length of $G$). .
        \item \(\varphi(G^{(i)}) = K^{(i)}\). In particular, homomorphic images and quotient groups of 
        solvable groups are solvable (of solvable length less than or equal to that of the 
        domain group).
        \item If \(N\) is normal in \(G\) and both \(N\) and \(G/N\) are solvable then so is \(G\).    
    \end{itemize}
\end{proposition}

\begin{theorem}\label{psol}
    All \(p\)-Gruppe is solvable.
\end{theorem}

\begin{theorem}
    Let \(p,q\) be prime with \(p<q\). Group of order \(pq\) is solvable (for \(p = q\) it is even abelian).
\end{theorem}

\begin{proof}
    Let \(G\) be a group of order \(pq\), wlog \(p < q\), \(s\) be the number of \(q\)-sylowsubgroup in G. We have \(s \mid p \wedge s = 1 \mod q \Rightarrow s = 1\). There is a unique \(q\)-sylowsubgroup \(U\) and \(U\) is normal in G. So \(U\) is group of order of \(q\) and \(G/U\) is group of order of \(p\). By theorem \ref{psol}, they are both solvable, so \(G\) is solvable.
\end{proof}

\begin{theorem}
    Let \(p,q,r\) be prime with \(p<q<r\). Group of order \(pqr\) is solvable.
\end{theorem}

\begin{proof}
    If \(G\) has only one sylowsubgroup for \(p,q,r\), then is the sylowsubgroup normal and the quotient group is solvable by proposition \ref{sol}. Then is \(G\) directly solvable by previous Theorem.\\
    Otherwise, assume there are more than \(q+1\)  \(q\)-sylowsubgroup and pq \(r\)-sylowsubgroup. So \(G\) have more than \((q+1)(q-1)\) element of order \(q\) and \(pq(r-1)\) element of order \(r\). Then \(G\) have at least \(pq(r-1)+(q+1)(q-1)+1 = pqr + q(q-p) > pqr = |G|\). This lead to a contradiction.
\end{proof}
