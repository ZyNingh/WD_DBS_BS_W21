\subsection{Composition series}
\begin{definition}[Simple group]
    A simple group is a nontrivial group whose only normal subgroup are the trivial group and the groups itself.
\end{definition}

\begin{definition}
    A composition series of a group G is a subnormal series of finite length\[1=H_{0}\unlhd H_{1}\unlhd \cdots \unlhd H_{n}=G\]
    with strict inclusions, such that each $H_i$ is a maximal proper normal subgroup of $H_{i+1}$.
\end{definition}
\begin{theorem}
    \begin{enumerate}
        \item The Schreier refinement theorem of group theory states that any two subnormal series of subgroups of a given group have equivalent refinements, where two series are equivalent if there is a bijection between their factor groups that sends each factor group to an isomorphic one.
        \item  The Jordan–Hölder theorem (named after Camille Jordan and Otto Hölder) states that any two composition series of a given group are equivalent.
    \end{enumerate}
\end{theorem}
\begin{remark}
    Noproof, whatever.
\end{remark}