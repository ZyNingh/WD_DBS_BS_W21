\subsection{Lineare Kongruenzen}
\begin{theorem}[Lineare Kongruenz]
    Eine lineare Kongruenz bezeichnet in der Zahlentheorie eine diophantische Gleichung in Form der Kongruenz

\[ax= b \mod m\]
Sei
\[ggT(a,m)= d\]
Diese Kongruenz hat genau dann Lösungen, wenn $d$ ein Teiler von $b$ ist:

\[d|b\]
Sei $r$ eine spezielle Lösung, dann besteht die Lösungsmenge aus $d$ verschiedenen Kongruenzklassen.
Die L\"osungen $x$ besitzen dann die Darstellung

\[x = r + t \cdot \frac{m}{d}, \ \  t\in \mathbb{Z}\]
\end{theorem}

\begin{theorem}[Chinesischer Restklassesatz]
    Schon
\end{theorem}