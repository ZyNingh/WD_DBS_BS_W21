\subsection{Die Relationale Algebra}
Wichtigste Beispiele:
\begin{itemize}
    \item Relationale Algebra
    \item Relationale-Kalk\"ul.
\end{itemize}

5 Grundoperationen der Relationalen Algebra:
\begin{itemize}
    \item Vereinigung: $R = S \cup T$
    \item Differenz: \(R = S - T\)
    \item Kartesisches Produkt: \(R = S \times T\)
    \item Selektion: \(R = \sigma_F(S)\)
    \item Projektion \(\pi_{A,B,...}(S)\)
\end{itemize}
\begin{remark}
    Die Kombinationen aus Selektion und kartesischen Produkt heiss Join.
\end{remark}
Eine Reihe n\"utzlicher Operationen lassen sich mit Hilfe der 5 Grundoperationen ausdr\"ucken:
\begin{itemize}
    \item Durchschnitt: $R=S\cap T$
    \item Quotient    $R = S \div T$
    \item Join    $R = S \bowtie T$
    \begin{itemize}
        \item Thera-Join $R \bowtie_{A\Theta B} S, \Theta\in \{=,\leq ,<,\geq, >,\not = \}$ (Denn: $R \bowtie_{A\Theta B} S = \sigma_{A\Theta B}(R\times S)$)
        \item Equi-Join $R \bowtie_{A= B}$
        \item Natural Join
    \end{itemize}
\end{itemize}