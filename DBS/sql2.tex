\subsection{Quantoren udn Subqueries in SQL}
\begin{example}[Subquery]
    \begin{verbatim}
        SELECT * FROM Kunde 
        WHERE EXISTS (SELECT ... FROM ... WHERE)
    \end{verbatim}
\end{example}
\begin{remark}
    \begin{itemize}
        \item In where-Klausel der Subquery auch Zugriff aud Relationen/Attribute der Hauptquery.
        \item Eindeutigkeit ggf. durch Aliasnamen f\"ur Relation. \begin{verbatim}
            SELECT * FROM kunde konstante1
            WHERE EXISTS ( SLEECT * 
                           FROM Kunde k2
                           WHERE k1.Adr = K2.Adr AND ...
                         )
        \end{verbatim}
        \item Existenz-Quantor ist realisiert mit dem Schl\"usselwort EXISTS.
        \item Term TRUE gdw. Ergebnis der Subquery nicht leer.
    \end{itemize}
\end{remark}

Es gibt keine direkt Unterst\"utzung in SQL von Allquantor. Aber es ist \"aquivalent \(\forall x:\varphi(x) \Leftrightarrow \neg \exists : \neg \neg \varphi(x)\). Also Notation in SQL:
\begin{verbatim}
    ... WHERE NOT EXISTS (SELECT ... FROM ... WHERE NOT ...)
\end{verbatim}

Direkt Subquery:\\
An jeder Stelle in der select- und where-Klausel, an der 
ein konstanter Wert stehen kann, kann auch eine Subquery 
(select...from...where...) stehen.
\begin{example}
    \begin{verbatim}
        SELECT Preis,
               Preis * (SELECT Kurs FROM Devisen
                        WHERE DName = 'US$' ) as USPreis
        FROM Waren Where ...
    \end{verbatim}
\end{example}

\begin{remark}
    Es gibt weitere Quantoren bei Standard-Vergleichung in WHERE.
    \begin{verbatim}
        - Ai = ALL (SELECT ... FROM ... WHERE ...)  // Allquantor
        - Ai = ANY (SELECT ... FROM ... WHERE ...)  // Ex.quantor
        - Ai = SOME (SELECT ... FROM ... WHERE ...) // Ex.quantor
    \end{verbatim}
    Es bedeutute :\(A_i \Theta \  \text{all}\ (\text{Subquery})\equiv \{...\mid \forall t \in \text{Subquery:}A_i \Theta t\}, \Theta\in\{=,<,\leq,>,\geq,<>\}\)
\end{remark}

\begin{remark}
    Es kann Subquery mit IN sein.
    \begin{verbatim}
        A_i [NOT] in ...
    \end{verbatim}
\end{remark}

\begin{example}Typische Fomr der Subquery:
    \begin{verbatim}
        SELECT ... FROM ... WHERE EXISTS (SELECT * FROM ...)
        SELECT ... FROM ... WHERE A <= ALL (SELECT B FROM ...)
        ... WHERE A <= (SELECT B FROM ... WHERE Schluessel = ...)
    \end{verbatim}
\end{example}

\subsection{Sortieren, Gruppieren und Views in SQL}
\subsubsection{Sortieren}
In SQL mit ORDER BY A1, A2, ...\\
Nach Attribut kann man ASC f\"ur aufsteigend(Default) order DESC f\"ur absteigend angeben.
\subsubsection{Aggregation}
Aggregation berechnet Eigenschaft ganzer Tupel-Menge. Es gibt folgendes Aggregatfunktionen in SQL:\begin{itemize}
    \item count
    \item sum
    \item avg
    \item max
    \item min
\end{itemize}
\begin{remark}
    \begin{itemize}
        \item NULL-Werte werden ignoriert(auch bei count).
        \item Eine Duplikatelimination kann erzwungen werden\begin{itemize}
            \item count(distinct KNname) z\"ahlt verschiedene Kunden
            \item count(all KName) z\"ahlt alle Eint\"age (ausser NULL)
            \item count(KName) ist identisch mit count(all KName)
            \item count(*) z\"ahlt die Tupel des Anfrageergebnisses
        \end{itemize}
    \end{itemize}
\end{remark}
\subsubsection{Gruppierung}
Syntax in SQL:
\begin{verbatim}
    SELECT          ...
    FROM            ...
    [WHERE          ...] 
    [group by A1,A2,...
     [HAVING        ...]]
    [ORDER BY       ...] 
\end{verbatim}
\subsubsection{View}
View ist eine Virtuelle Relation, die f\"ur den Benutzer aus wie eine Relation sieht. Aber die Relation ist nicht real existent/gespeichert. Definition und L\"oschen in SQL:
\begin{verbatim}
    CREATE [OR REPLACE] view VName [(A1,A2,...)] AS SELECT ...
    drop view VName
\end{verbatim}
\begin{remark}
    \begin{itemize}
        \item Automatisch sind in dieser View alle Tupel der Basisrelation, die die Selecktionsbedingung erf\"ullen.
        \item In Wirklichkeit wird lediglich die View-definition in die Anfrage eingesetzt und dann ausgewertet.
        \item Bei Updates in der Basisrelation \"andert sich auch die virtuelle Relation. Umgekehrt k\"onnen (mit Einschr\"ankungen) auch \"Anderungen an der View durchgef\"uhrt werden, die sich auf die Basisrelation auswirken.
        \item Views sind ein wichtiges Strukturierungsmittel f\"ur Anfragen und die gesamte Datenbank.
        \item Selektion, Projektion, Kreuzprodkt, Join, Vereinigung, Differenz, Schnitt, Gruppieren, Aggregation und Subqueries sind erlaubt in VIEW.
        \item Sortieren ist nicht erlaubt.
    \end{itemize}
\end{remark}

FIXME: Zhe yi duan shi sha ya .

\subsubsection{Rechtvergabe}
Syntax:\begin{verbatim}
    GRANT Rechteliste
    ON Relation         //bzw. View
    TO Benutzerliste
    [with grant option]
\end{verbatim}
\begin{remark}
    M\"oglichkeit:\begin{itemize}
        \item Rechteliste: \begin{itemize}
            \item ALL [privileges]
            \item SELECT, INSERT, DELETE (mit Kommas sep.)
            \item UPDATE (optional in Klammern: Attributnamen)
        \end{itemize}
        \item Benutzerliste:\begin{itemize}
            \item Benutzernamen (mit Passwort identifizert)
            \item TO PUBLIC
        \end{itemize}
        \item R\"ucknahme von Rechten:
        \begin{verbatim}
            REVOKE Rechteliste
            ON Relation
            FROM Benutzerliste
            [RESTRICT] Abbruch, falls Recht bereits weitergegeben
            [CASCADE] ggf. Propagierung der Revoke-Anweisung
        \end{verbatim}
    \end{itemize}
\end{remark}
