\subsection{Quantoren udn Subqueries in SQL}
\begin{example}[Subquery]
    \begin{verbatim}
        SELECT * FROM Kunde 
        WHERE EXISTS (SELECT ... FROM ... WHERE)
    \end{verbatim}
\end{example}
\begin{remark}
    \begin{itemize}
        \item In where-Klausel der Subquery auch Zugriff aud Relationen/Attribute der Hauptquery.
        \item Eindeutigkeit ggf. durch Aliasnamen f\"ur Relation. \begin{verbatim}
            SELECT * FROM kunde konstante1
            WHERE EXISTS ( SLEECT * 
                           FROM Kunde k2
                           WHERE k1.Adr = K2.Adr AND ...
                         )
        \end{verbatim}
        \item Existenz-Quantor ist realisiert mit dem Schl\"usselwort EXISTS.
        \item Term TRUE gdw. Ergebnis der Subquery nicht leer.
    \end{itemize}
\end{remark}

Es gibt keine direkt Unterst\"utzung in SQL von Allquantor. Aber es ist \"aquivalent \(\forall x:\varphi(x) \Leftrightarrow \neg \exists : \neg \neg \varphi(x)\). Also Notation in SQL:
\begin{verbatim}
    ... WHERE NOT EXISTS (SELECT ... FROM ... WHERE NOT ...)
\end{verbatim}

Direkt Subquery:\\
An jeder Stelle in der select- und where-Klausel, an der 
ein konstanter Wert stehen kann, kann auch eine Subquery 
(select...from...where...) stehen.
\begin{example}
    \begin{verbatim}
        SELECT Preis,
               Preis * (SELECT Kurs FROM Devisen
                        WHERE DName = 'US$' ) as USPreis
        FROM Waren Where ...
    \end{verbatim}
\end{example}

\begin{remark}
    Es gibt weitere Quantoren bei Standard-Vergleichung in WHERE.
    \begin{verbatim}
        - Ai = ALL (SELECT ... FROM ... WHERE ...)  // Allquantor
        - Ai = ANY (SELECT ... FROM ... WHERE ...)  // Ex.quantor
        - Ai = SOME (SELECT ... FROM ... WHERE ...) // Ex.quantor
    \end{verbatim}
    Es bedeutute :\(A_i \Theta \  \text{all}\ (\text{Subquery})\equiv \{...\mid \forall t \in \text{Subquery:}A_i \Theta t\}, \Theta\in\{=,<,\leq,>,\geq,<>\}\)
\end{remark}

\begin{remark}
    Es kann Subquery mit IN sein.
    \begin{verbatim}
        A_i [NOT] in ...
    \end{verbatim}
\end{remark}

\begin{example}Typische Fomr der Subquery:
    \begin{verbatim}
        SELECT ... FROM ... WHERE EXISTS (SELECT * FROM ...)
        SELECT ... FROM ... WHERE A <= ALL (SELECT B FROM ...)
        ... WHERE A <= (SELECT B FROM ... WHERE Schluessel = ...)
    \end{verbatim}
\end{example}