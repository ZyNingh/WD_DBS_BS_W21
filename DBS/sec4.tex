\subsection{Realtionen-Kalk\"ul}
\begin{itemize}
    \item Relatione Algebra ist prozedurale Sprache:\\
    Ausdruck gibt an, unter Benutzung welcher Operation deas Ergebnis berechnet werden soll
    \item Relationen-Kalk\"ul ist deklarative Sprache:\\
    Ausdruck bereischreibt, welche Eigenschaft die Tupel der Ergebnisrelation haben m\"ussen ohne eine Berechnungsprozedur daf\"ur anzugeben.
    \item Es gibt zwei vershiedene Ans\"atze:
    \begin{itemize}
        \item Tupelkalk\"ul: Variable sind vom Typ Tupel.
        \item Breichskalk\"ul: Variablen haben einfachen Typ.
    \end{itemize}
\end{itemize}
\begin{remark}
    Hier gibt es eine relativ lange Teil \"uber Mahtematik. Ich habe gar kein Interesse daran. Etwas \"uber Formel, Sytax, Sematik, Interpretation(Belegung von Variablen)
\end{remark}

\subsubsection{Tupelkalk\"ul}
Man arbeitet mit \begin{itemize}
    \item Tupelvaribalen:  \(t\)
    \item Formeln: \(\varphi(t)\)
    \item Ausdr\"ucken: \(\{t\mid \varphi(t)\}\)
\end{itemize}
\begin{definition}[Definitertes Schema von Tupelvaribalen]
    \begin{itemize}
        \item Schema(\(T\)) = (\(A_1:D_1,A_2:D_2,\cdots\))
        \item Schema(\(T\)) = \(R\).
    \end{itemize}    
\end{definition}
\begin{remark}
    \(t[A]\) oder \(t.A\) f\"ur einen Attributnamen \(A\in \)Schema(\(t\)).
\end{remark}

\subsubsection{Bereichskalk\"ul}
\begin{definition}
    Ein Ausdruck hat die Form:
    \[\{x_1,x_2,\cdots,\mid \varphi(x_1,x_2,\cdots)\}\]
    Atome haben die Form:\begin{itemize}
        \item \(R_1(x_1,x_2,\cdots)\): Tupel \((x_1,x_2,\cdots)\) tritt in Relation \(R\) auf.
        \item x\(\Theta\): \(x,y\) Bereichsvariblen bzw. Konstant.\(\Theta \in \{=,<,\leq,>,\geq,\not =\}\)
    \end{itemize}
\end{definition}