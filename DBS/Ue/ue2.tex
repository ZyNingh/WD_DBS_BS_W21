
\section{\"Ubungsblatt 2}
Thema: \textbf{Schl\"ussel, SQL-DDL}
\subsection{Aufgabe 2-1}
\subsubsection{a)}
    Einf\"ugen von Duplikaten wird verweigert.
\subsubsection{b)}
    Erm\"glicht effizientere \"Ubungspr\"ufung von Sch\"ussel/Fremdsch\"usselbeziehung.
\subsubsection{c)}
    Zu jedem Attributwert in abh\"angiger Relation muss auch entsprechenden Schl\"usselwert in referenzierter Relation existieren. (No dangling references)
\subsubsection{d)}
    \begin{itemize}
        \item Nein, Referenzielle Integrit\"at verletzt.
        \item Nein, Schl\"usseleigenschaft verletzt.
        \item Gut.
    \end{itemize}

\subsection{Aufgabe 2-2}
\begin{verbatim}
a)
create table L
(
    lnr integer primary key,
    lname varchar(100) not null,
    sitz varchar(100)
);

create table T
(
    tnr integer primary key,
    tname varchar(100) not null,
    farbe varchar(100),
    gewicht float,
    preis decimal(8,4) check (preis > 0)
);

create table P
(
    pnr integer primary key,
    pname varchar(100) not null,
    ort varchar(100)
);


create table LTP
(
    lnr integer,
    tnr integer,
    pnr integer,
    menge integer check (menge > 0),
    primary key (lnr, tnr, pnr),
    foreign key (lnr) references L(lnr),
    foreign key (tnr) references T(tnr),
    foreign key (pnr) references P(pnr)
)

b)
alter table P add
(
    status integer default 5
)

c)
alter table T modify gewicht float check(gewicht >0)

d)
alter table P drop ort

e)
drop table LTP;
drop table L;
drop table T;
drop table P

\end{verbatim}