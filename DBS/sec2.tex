
\section{Relation}
\subsection{Das Relationale Modell}
\begin{definition}[Domain]
    Domain ist ein Wertbereich, der endliche oder unendlich sein kann.
\end{definition}

\begin{definition}[Kartesisches Produkt]
    Kartesisches Produkt von $k$ Menge ist Menge von allen m\"oglichen Kombinationen der Elemente der Menge.
\end{definition}

\begin{definition}[Relation]
    Relation $R$  ist Teilmenge des kartesischen Produktes von $k$\\ Domains $D_1,\cdots, D_k$ \[R\subseteq D_1\times D_2 \times\cdots \times D_k\]
\end{definition}
\begin{remark}
    \begin{itemize}
        \item Die Relation kann endlich oder unendlich sein. Aber in Datenbanksysteme haben wir nur endliche Relation.
        \item Die Anzahl der Tupel einer Relation heisst Kardinalit\"at $|\cdot|$. (Tupel is Elemente der Relation.)
        \item Die einzelnen Domains lassen sich als Spalten einer Tabelle verstehen und werden als Attribute bezeichnet.
        \item Die Reihenfolge der Tupel spielt keine Rolle. Reihenfolge der Attribute ist von Bedeutung.
    \end{itemize}
\end{remark}

\begin{definition}[Relations-Schema(Alternative Definition in DBS)]
    Relation ist Auspr\"agung eines Realtion-Schemas.
    \begin{itemize}
        \item Geordnetes Relationenschema\[R = (A_1:D_1,\cdots , A_k:D_k)\]
        \item Dom\"anen-Abbildung (ungeordnetes Rel.-Sch.) \[R = \{A_1,\cdots,A_k\}\  \text{ mit }\  dom (A_i) = D_i, i \leq i \leq k\]
    \end{itemize}
\end{definition}
\begin{remark}
    \begin{enumerate}
        \item[Vor:](geordnetes RS) Prägnanter aufzuschreiben.
        \item[Nach:](geordnetes RS) Einschr\"ankungen bei logischer Datenunabh\"anigigkeit: Einf\"ugung neuer Attribute ist nur am Ende m\"oglich.  
        \item[Ungeordnetes RS:] Reihenfolge der Spalte(Attribute) ist irrelevant. 
    \end{enumerate}
\end{remark}

\begin{remark}
    Begriff:
    \begin{itemize}
        \item Relation: Auspr\"agung eines Relationenschema.
        \item Datenbankscheme: Menge von Relationenschema.
        \item Menge von Relation(Auspr\"agungen).
    \end{itemize}
\end{remark}

%S.12

\begin{remark}
    Relation sind Menge von Tupel. Dann ist Reihenfolge der Tupel irrelevant. Und Duplikate kann nicht auftreten.
\end{remark}

\begin{definition}[Schl\"ussel]
    Ein Schlüssel dient in einer relationalen Datenbank dazu, die Tupel einer Relation eindeutig zu identifizieren, sie zu nummern.
\end{definition}
\begin{remark}
    \begin{itemize}
        \item Ein/mehrere Attribute als Schl\"ussel kennzeichen.
        \item Oft ist ein einzelnes Attribut nicht ausreichend, um die Tupel eindeutig zu identifizieren.
        \item Das muss eindeutig sein.
    \end{itemize}
\end{remark}
\begin{definition}[Schl\"ussel, formale]
    Eine Telimeng $S$ der Attribute eines Relationschemas $R$ ($S\subset R$) heisst Schul\"ussel, wenn gilt:
    \begin{enumerate}
        \item[Eindeutigkeit:] Keine Auspr\"agung von $R$ kann zwei verschiedene Turpe enthalten, die sich in allem Attributen von $S$ gleichen.($\forall \ \text{Auspr\"agung} \  r\forall t_1,t_2 \in r:t_1\not = t_2 \Rightarrow t_1[S] = t_2[s].$)
        \item[Minimalit\"at:] Ex existiert keine echte Teilmenge $T\subsetneq S$, die breit die Bedigung der Eindeutigkeit erf\"ullt. ($\forall$ Attributmenge $T$, die Eindeutigkeit erf\"ullen, gilt: $T\subset S \Rightarrow T=S$.)
    \end{enumerate}
\end{definition}
\begin{remark}
    \begin{itemize}
        \item Eine Menge $S\subseteq R$ heisst Superschl\"ussel, wenn sie die Eindeutigkeitseingenschaft erf\"ullt.
        \item In der Mathematik wird allgemein eine Menge $M$ als minimale Menge bez\"uglich einer Eigenschaft $B$ bezeichnet, wenn es keine echte Teilmenge von $M$ gibt, die ebenfalls $B$ erf\"ullt.
        \item Damit k\"onnen wir auch definiren: Ein Schl\"usseln ist ein minimaler Superschl\"ussel.
        \item Minimalit\"at bedeutet nicht: Schl\"ussel mit den wenigsten Attributen, sondern keine \"uberfl\"ussigen Attribute sind enthalten.
        \item Manchmal gibt es mehrere verschiedene Schl\"ussel.
    \end{itemize}
\end{remark}
Man w\"ahlte einen dieser Kandidaten aus als sogenannter Prim\"arschl\"ussel.
\begin{definition}[Fremdschl\"ussel]
    Attribute das auf einen Schl\"ussel einer anderen Relation verweist, heisst Fremdschl\"ussel.
\end{definition}
\begin{remark}
    Die Eindeutigkeit bezieht sicht nicht auf die aktuelle Auspr\"agung eine Relation. Sondern immer auf die Semantik der rearlen Welt.
\end{remark}

%S.23

