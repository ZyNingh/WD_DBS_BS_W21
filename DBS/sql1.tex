\section{SQL}
\subsection{Tabelledefinition in SQL}
\begin{verbatim}
    CREATE TABLE name       // name Name der Relation
    (
        a1 d1 c1,           // definition des ersten Attributs
        a2 d2 c2,
        ....
        ak dk ck            // definition des Attributs Nr. k
    )
\end{verbatim}
Hierbei bedeuten
\begin{itemize}
    \item ai der name des Attributs Nr. i
    \item di der Typ (die Domain) des Attributs
    \item ci ein optionaler Constraint f\"ur das Attribut
\end{itemize}
Wirkung: Definition eines Realtionschemas mit einer leeren Realtion als Auspr\"agung.
\\
Der SQL-Standard kennt u.a. folgende Datentypen:
\begin{itemize}
    \item integer oder auch integer4, integer
    \item smallint oder integer2
    \item float(p) oder auch float
    \item decimal(p,q) und numeric(p,q) mit p Stellen, davon q Nachkommast.
    \item character(n), char(n) f\"ur Strings fester L\"ange n
    \item character varying(n), varchar(n): variable Strings
    \item date, time, timestamp f\"ur Datum und Zeit
\end{itemize}
Einfache Zus\"atze (Integrit\"atsbedingungen) k\"onnen unmittelbar hinter einer Attributdefinition stehen:
\begin{itemize}
    \item not null
    \item primary key
    \item unique
    \item references t1(a1)
    \item default w1: Wert w1 ist Default, wenn unbesetzt.
    \item check f
\end{itemize}
Zus\"atze, die keinem einzelnen Attribut zugeordnet sind, stehen mit Komma abgetrennt in extra Zeilen
%S.27
\begin{itemize}
    \item primary key(A1,A2,...)
    \item unique (A1,A2,...)
    \item foreign key (A1,A2,...) references t1 (B1,B2,...)
    \item check f
\end{itemize}
\begin{remark}
    SQL ist case-insensitiv.
\end{remark}
L\"osche eines Tupels in B Referenzen nicht m\"oglich.
Es gibt aber verschiedene Zus\"atze:.\\
foreign key(a5,a6) references B(b1,b2)
\begin{itemize}
    \item on delete cascade
    \item on update cascade
    \item on delete set null
\end{itemize}
FIXME:\\
Weitere DDL(Data Definition Language):
\begin{verbatim}
    drop table n1  // RS n1 wird mit allen evtl.
                   / /vorhandenen Tupeln geloescht
    alter table n1 add (a1 d1 c1, a2 d2 c2, ...)
    alter table drop (a1, a2, ...)
    alter table modify (a1 d1 c1, a2 d2 c2, ...)
\end{verbatim}

\subsection{Anfragen}
Grundform einer Anfrage:
\begin{verbatim}
    Projektion  -> SELECT <List von Attributsnamene bzw. *>
    Kreuzprodkt -> FROM   <ein oder mehrere Relationennamen>
    Selektion   -> [WHERE <Bedigung>]
\end{verbatim}
Mengenoperationen:
\begin{verbatim}
    SELECT ... FROM ... WHERE ... 
    UNION
    SELECT ... FROM ... WHERE ...
\end{verbatim}

\begin{remark}
    \begin{itemize}
        \item SQL ist relational vollst\"andig.
        \item Duplikatelimination nur, wenn durch das Schlüsselwort DISTINCT explizit verlang:
              \begin{verbatim}
            SELECT * FROM ...           -- Keine Projektion
            SELECT A1, A2, ... FROM ... -- Projektion ohne
                                    -- Duplikatelimination
            SELECT DISTINCT A1, A2, ... -- Projektion mit
                                    -- Duplikatelimination
        \end{verbatim}
        \item Mann kann Schreibarbeit sparen, indem man den Relationen lokal kurze Namen zuweist(Alias-Namen):
              \begin{verbatim}
            SELECT m.Name, a.Name
            FROM Mitarbeiter m, Abteilung a
            WHERE ...
        \end{verbatim}
        \item Where\begin{itemize}
                  \item Vergleichsoperatoren: \(=, <, \leq, > , \geq ,<>\)
                  \item Test auf Wert undefinirt:  A IS NULL / IS NOT NULL
                  \item Inexakter Stringvergleich: A LIKE 'Datenbank\%'
                        \begin{itemize}
                            \item \% steht f\"ur einen beliebig belegbaren Teilstring
                            \item \_ steht f\"ur genau ein einzelnes frei belegbares Zeichen
                        \end{itemize}
                  \item A1 IN (2, 3, 4, 5, 12, 13)
              \end{itemize}
    \end{itemize}

\end{remark}

Join: Normalerweise wird der Join wie bei der relationalen Algebra als Selektionsbedingung über dem kartesischen Produkt formuliert.
Folgende Anfrage sind m\"oglich in SQL:
\begin{verbatim}
    SELECT * FROM Mitarbeiter m, Abteilung a WHERE m.ANr = a.ANr
    SELECT * FROM Mitarbeiter m JOIN Abteilung a on a.ANr = m.ANr
    SELECT * FROM Mitarbeiter JOIN Abteilung using (ANr)
    SELECT * FROM Mitarbeiter natural JOIN Abteilung
\end{verbatim}
TODO:OUTER JOIN. bu ru du yi ba, qi mo kao bu hui kao outer join.

\"Anderungs-Operationen:\\
Grunds\"atzlich unterscheiden wir:
\begin{verbatim}
    INSERT:  Einfuegen von Tupeln in eine Relation
    DELETE:  Loeschen von Tupeln aus einer Relation
    UPDATE:  Aedern von Tupeln einer Relation
\end{verbatim}
\begin{remark}
    Usage:
    \begin{itemize}
        \item UPDATE: \begin{verbatim}
            UPDATE Realtion
            SET attribut1 = ausdruck1
                [,...,
                attributn = ausdruckn]
            [WHERE bedingung]
        \end{verbatim}
              \begin{itemize}
                  \item UPDATE-Operationen k\"onnen zur Verletzung von Integrit\"atsbedingungen f\"uhren: Abbruch der Operation mit Fehlermeldung.
              \end{itemize}
        \item DELTEE: \begin{verbatim}
            DELETE FROM relation
            [WHERE bedingung]
        \end{verbatim}
              \begin{itemize}
                  \item L\"oscht alle Tupel, die die Bedingung erfüllen
                  \item Ist keine Bedingung angegeben, werden alle Tupel
                        gel\"oscht
                  \item Abbruch der Operation, falls eine Integrit\"atsbedingung verletzt w\"urde (z.B. Fremdschlüssel ohne cascade)
              \end{itemize}
        \item INSERT:\begin{verbatim}
            INSERT INTO relation (attribut1, attribut2,...)
            VALUES (konstante1, konstante2, ...)
            or
            INSERT INTO relation 
            VALUES (konstante1, konstante2, ...)
            or
            INSERT INTO relation [(attribut1 , ...)]
            ( SELECT ... FROM ... WHERE ... )
        \end{verbatim}
        \begin{itemize}
            \item Ist die optionale Attributliste hinter dem Relationennamen angegeben, dann
                  \begin{itemize}
                      \item k\"onnen unvollständige Tupel eingef\"ugt werden: Nicht aufgef\"uhrte Attribute werden mit NULL belegt.
                      \item werden die Werte durch die Reihenfolge in der
                            Attributsliste zugeordnet.
                  \end{itemize}
            \item Ist die optionale Attributliste hinter dem Relationennamen
                  angegeben, dann\begin{itemize}
                      \item k\"onnen unvollst\"andige Tupel eingef\"ugt werden:
                            Nicht aufgef\"uhrte Attribute werden mit NULL belegt
                      \item werden die Werte durch die Reihenfolge in der
                            Attributsliste zugeordnet. (mangelnde Datenunabh\"angigkeit!)
                  \end{itemize}
            \item Wirkung zum Einf\"ugen berechneter Tupel:
                  \begin{itemize}
                      \item Alle Tupel des Ergebnisses der SELECT-Anweisung werden in die Relation eingef\"ugt.
                      \item Die optionale Attributliste hat dieselbe Bedeutung wie bei der entsprechenden Ein-Tupel-Operation.
                      \item Bei Verletzung von Integrit\"atsbedingungen (z.B. Fremdschl\"ussel nicht vorhanden) wird die Operation nicht ausgeführt (Fehlermeldung).
                  \end{itemize}
        \end{itemize}
    \end{itemize}
\end{remark}